% Chapter Template

\chapter{no longer needed} % Main chapter title

\label{Chapter7} % Change X to a consecutive number; for referencing this chapter elsewhere, use \ref{ChapterX}

%----------------------------------------------------------------------------------------
%	SECTION 1
%----------------------------------------------------------------------------------------
One of the aims of this project was to create hardware that is capable of successfully spoofing a GPS receiver. From the results it can be seen that the method of
spoofing using signals generated from coordinates was successful. There were issues that impacted the results at times, however, these were overcome to produce good
results in both static and dynamic spoofing. The hardware that was used for this is capable of performing any spoofing attack, however the
software was not capable to produce the results at times, where anti-spoofing was at play.
Overall the accuracy of the position gathered from the GPS receivers was poorer than expected, although
to be able to spoof anything at all was considered a success. Given these results were gathered in a controlled environment more care will need to be taken if
performing these experiments in the field. 

\bigskip

The project was able to provide a foundation for which further research into GNSS attack and defence strategies can be
performed. SDR devices are prevalent in EW research due to their combination of hardware level performance and software level flexibility. This will allow for further
research into other areas of EW.
There were no results gathered from performing Meaconing attacks since the GPS signal reception was unsuccessful and therefore signal capture was unavailable.

Through this thesis it has been shown that with relatively low amounts of technical knowledge successful spoofing attacks can be performed. Some basic computer literacy
was required in order to install and configure the software needed to perform the attacks. The simplest way of using the software is with Linux which does add a level of
difficulty.
Moving forward into the future the software could be given a graphical interface and integrated into one software package, or simplify the transmission interface and use
pre compiled binaries.