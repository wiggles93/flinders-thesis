% Chapter Template

\chapter{Conclusion} % Main chapter title

\label{Chapter6} % Change X to a consecutive number; for referencing this chapter elsewhere, use \ref{ChapterX}

%----------------------------------------------------------------------------------------
%	SECTION 1
%----------------------------------------------------------------------------------------

One of the aims of this project was to create hardware that is capable of successfully spoofing a GPS receiver. From the results it can be seen that the method of
spoofing using signals generated from coordinates was successful. There were issues that impacted the results at times, however, these were overcome to produce good
results in both static and dynamic spoofing. The hardware that was used for this is capable of performing any spoofing attack, however the
software was not capable to produce the results at times, where anti-spoofing was at play.
Overall the accuracy of the position gathered from the GPS receivers was poorer than expected, although
to be able to spoof anything at all was considered a success. Given these results were gathered in a controlled environment more care will need to be taken if
performing these experiments in the field. 

\bigskip

The project was able to provide a foundation for which further research into GNSS attack and defence strategies can be
performed. SDR devices are prevalent in EW research due to their combination of hardware level performance and software level flexibility. This will allow for further
research into other areas of EW.
There were no results gathered from performing Meaconing attacks since the GPS signal reception was unsuccessful and therefore signal capture was unavailable.

Through this thesis it has been shown that with relatively low amounts of technical knowledge successful spoofing attacks can be performed. Some basic computer literacy
was required in order to install and configure the software needed to perform the attacks. The simplest way of using the software is with Linux which does add a level of
difficulty.
Moving forward into the future the software could be given a graphical interface and integrated into one software package, or simplify the transmission interface and use
pre compiled binaries.

\section{Future Work}
All of the tests that have been conducted for use in this thesis have either been generating a signal based off a predetermined location (coordinate) or a predetermined
path (group of connected coordinates). None of these options have a mechanism for position feedback. For example you cannot create a signal that has a linear offset from
the actual position using these methods. This is something that could be achieved through combination and modification of the existing open source projects used in this
thesis. This would allow for properly real-time GPS spoofing, and could be extended further to have intelligent algorithms.

Current methods, as detailed in Chapter \ref{Chapter4}, requires downloading the Ephemeris for the date and time of the proposed spoofing attack. This makes real-time spoofing
attacks not possible since there is a delay between the current time and the associated ephemeris. A potential fix for this could be to perform some analysis on the
orbits of each satellite within the constellation over an arbitrary period of time to create a machine learning algorithm. This algorithm would be able to predict the
location of the satellites ahead of time, thus binaries could be compiled for future attacks, or could be used for real-time attacks without the need for downloading/
retrieving the ephemeris from legitimate sources.

The logging program could be extended to provide real time feedback on position and signal quality. This would be achieved by implementing the serial connection within
the Python code itself rather than relying on the logging function of screen.
Extended to work on anti-spoofing for PhD. If there was more time for this project, a software based anti spoofing program would have been easy to implement based
on the $\frac{C}{N_0}$ ratio alone. This could be implemented in conjunction with real time graphing as mentioned above.

The hardware used for this project was successfully used for spoofing attacks, however, the software must be altered in order to achieve the accurate results that were
expected of the project. However ensuring software capabilities was out of the scope of this thesis.