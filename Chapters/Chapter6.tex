% Chapter Template

\chapter{Discussion} % Main chapter title

\label{Chapter6} % Change X to a consecutive number; for referencing this chapter elsewhere, use \ref{ChapterX}
This section will discuss the results gathered from the previous section \ref{Chapter5}. Any broad findings or trends will be noted and explanations of why results are
the way they are.

%----------------------------------------------------------------------------------------
%	SECTION 1
%----------------------------------------------------------------------------------------

\section{Results}
Initially it was decided that the locations that would be spoofed would be locations that are not easily accessible as a means of proving the validity of the results.
This was made easier due to the ongoing pandemic. Although this idea was later changed to make it easier to compare with real signals if needed.

Using the developed workflow, it was simple to create and deploy spoofing attacks for any location and at any elapsed time within minutes. This includes static or dynamic
spoofing attacks. Not enough time was invested into the reception of GNSS signals and thus meaconing attacks were not performed. The GPS-SDR-sim program was able to
estimate and simulate the ionic effects. While it is clear that this method resulted in successful spoofing attacks, the hypothesis is that having a received signal that includes ionic delays
and other interference would make it easier to fool more complicated receivers. Further testing and investigations should be carried out to find the answer to this. 

The results from the static Antarctica spoof attack, specifically the carrier to noise graphs show a high number of unique SVIDs. Based on the number of satellites in the
GPS constellation this must be an error. Although there was no explanation for the error since the resultant location was correct. One hypothesis was that the compilation
software added some incorrect data. Another is the prevalence of SBAS satellites. There was a known issue for some time with the Faraday cage where stray EM waves were
getting in. This would explain the consistency in which SVID 193 appears. From \todo{ref} it can be seen that SVID 193 is a QZSS satellite.

SDRs are very sensitive to the processing pipeline of the computer that they are connected to. It is not uncommon to receive an under run error (where 'U' is printed to the
output) when transmitting a signal with high SPS (samples per second). This error is caused by the host computer not being able to feed the data to the radio quickly
enough.

This was solved by increasing the UDP buffer size manually to at least the sample rate, and over for better results. From limited experimentation this completely resolved the issue, and opened up
opportunities to perform spoofing with low powered embedded devices like the Raspberry Pi single board computer.

\section{Future Work}
All of the tests that have been conducted for use in this thesis have either been generating a signal based off a predetermined location (coordinate) or a predetermined
path (group of connected coordinates). None of these options have a mechanism for position feedback. For example you cannot create a signal that has a linear offset from
the actual position using these methods. This is something that could be achieved through combination and modification of the existing open source projects used in this
thesis. This would allow for properly real-time GPS spoofing, and could be extended further to have intelligent algorithms.

Current methods, as detailed in \ref{Chapter4}, requires downloading the Ephemeris for the date and time of the proposed spoofing attack. This makes real-time spoofing
attacks not possible since there is a delay between the current time and the associated ephemeris. A potential fix for this could be to perform some analysis on the
orbits of each satellite within the constellation over an arbitrary period of time to create a machine learning algorithm. This algorithm would be able to predict the
location of the satellites ahead of time, thus binaries could be compiled for future attacks, or could be used for real-time attacks without the need for downloading/
retrieving the ephemeris from legitimate sources.

Extended to work on anti-spoofing for PhD.

The logging program could be extended to provide real time feedback on position and signal quality. This would be achieved by implementing the serial connection within
the Python code itself rather than relying on the logging function of screen.

The hardware used for this project was successfully used for spoofing attacks, however, the software must be altered in order to achieve the accurate results that were
expected of the project. However ensuring software capabilities was out of the scope of this thesis.