% Chapter Template

\chapter{Methodology} % Main chapter title

\label{Chapter3} % Change X to a consecutive number; for referencing this chapter elsewhere, use \ref{ChapterX}
%\medskip
Summary of testing:
\begin{itemize}
    \item Used commercial GPS receiver to check received SNR values \\ Antennae used:
    \begin{itemize}
        \item passive patch antenna
        \item active antenna
        \item Log-periodic antenna
    \end{itemize}
    This resulted in a preliminary finding that the active antenna improved signal to noise ratio by 8dB. \todo{Update with more formal findings}
    \item Use SDR + GNSS-SDR to receive similar signals and compare SNR values
    \item Use stored text file of gps data and connect Tx to Rx antenna port (via attenuator) to check if programs are manipulating the data as they should be.
    \item Use stored txt file of gps data (received from commercial receiver) and transmit gps signal using log periodic antenna to see if phone GPS can be spoofed into thinking it is in another geographical location.
\end{itemize}
\medskip
All of the software packages used are free and opensource.
List of Software packages used:
\begin{itemize}
    \item GNURadio
    \item GNSS-SDR
    \item GNSS-SDR-monitor
    \item gps-sdr-sim
\end{itemize}

From the desktop version of Google Earth Pro navigation paths can be created. These paths can then be use to generate a GPS signal to replicate that path using the
GPS-SDR-SIM program \cite{RN42}. It should be noted that the KML file format is not suitable for use with this program and must be converted into NMEA GGA stream format.

\textbf{It would be great to be able to perform a meaconing attack. This requries real time reception and rebroadcast with a higher power. The extra time delay
makes the spoofing victim believe they are elsewhere. This should be a search term.}

%----------------------------------------------------------------------------------------
%	SECTION 1
%----------------------------------------------------------------------------------------

\section{Testing Methodology}

In this section the method of creating a GPS signal spoofing device is detailed. The main part of this project is the upgradability of the SDR platform
and in particular the USRP N210 SDR. As previously mentioned the benefit of using an SDR over using an ASIC is that with a new version of software 
new capacilities are availeble to the device. This could be benefitial to either the spoofer or spoof defense. 

%-----------------------------------
%	SUBSECTION 1
%-----------------------------------
\subsection{Subsection 1}

Nunc posuere quam at lectus tristique eu ultrices augue venenatis. Vestibulum ante ipsum primis in faucibus orci luctus et ultrices posuere cubilia Curae; Aliquam erat volutpat. Vivamus sodales tortor eget quam adipiscing in vulputate ante ullamcorper. Sed eros ante, lacinia et sollicitudin et, aliquam sit amet augue. In hac habitasse platea dictumst.

%-----------------------------------
%	SUBSECTION 2
%-----------------------------------

\subsection{Subsection 2}
Morbi rutrum odio eget arcu adipiscing sodales. Aenean et purus a est pulvinar pellentesque. Cras in elit neque, quis varius elit. Phasellus fringilla, nibh eu tempus venenatis, dolor elit posuere quam, quis adipiscing urna leo nec orci. Sed nec nulla auctor odio aliquet consequat. Ut nec nulla in ante ullamcorper aliquam at sed dolor. Phasellus fermentum magna in augue gravida cursus. Cras sed pretium lorem. Pellentesque eget ornare odio. Proin accumsan, massa viverra cursus pharetra, ipsum nisi lobortis velit, a malesuada dolor lorem eu neque.

%----------------------------------------------------------------------------------------
%	SECTION 2
%----------------------------------------------------------------------------------------

\section{Main Section 2}

Sed ullamcorper quam eu nisl interdum at interdum enim egestas. Aliquam placerat justo sed lectus lobortis ut porta nisl porttitor. Vestibulum mi dolor, lacinia molestie gravida at, tempus vitae ligula. Donec eget quam sapien, in viverra eros. Donec pellentesque justo a massa fringilla non vestibulum metus vestibulum. Vestibulum in orci quis felis tempor lacinia. Vivamus ornare ultrices facilisis. Ut hendrerit volutpat vulputate. Morbi condimentum venenatis augue, id porta ipsum vulputate in. Curabitur luctus tempus justo. Vestibulum risus lectus, adipiscing nec condimentum quis, condimentum nec nisl. Aliquam dictum sagittis velit sed iaculis. Morbi tristique augue sit amet nulla pulvinar id facilisis ligula mollis. Nam elit libero, tincidunt ut aliquam at, molestie in quam. Aenean rhoncus vehicula hendrerit.