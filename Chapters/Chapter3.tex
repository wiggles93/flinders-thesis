% Chapter Template

\chapter{Literature Review} % Main chapter title

\label{Chapter3} % Change X to a consecutive number; for referencing this chapter elsewhere, use \ref{ChapterX}

%----------------------------------------------------------------------------------------
%	SECTION 1
%----------------------------------------------------------------------------------------

\section{Introduction}
This section will go through the existing literature around the topic of GNSS spoofing and anti-spoofing. References were chosen based on their relevance in terms of
content as well as age. There have been some significant performance increases in technologies over the past 10 years.
This literature review will focus mainly on GNSS spoofing methods, with emphasis on the GPS constellation, and on potential anti-spoofing methods. While development of
anti-spoofing algorithms is not an aim of this thesis, knowledge of such methods allow for understanding potential attack vectors.

\section{Literature Review}
How GPS works has been covered in depth over the past few decades. The operating pricinpals for GPS and other GNSS systems is well known and was covered in the previous
chapter \ref{Chapter2}. 

From literature the consensus is that the same thing that has made GPS ubiquitous with navigation and 
positioning has also made it a simple target for exploitation and manipulation, that is the workings
of the infrastructure are well known and public and are transparent and predictable \cite{RN7} \cite{RN4}. This is problematic since this infrastructure
is seen as a critical service by many industries including utility management, healthcare/ emergency services and security.
Having such a system so prone to threats is not ideal. The further development of SDR platforms has driven down the cost of launching
such exploits. Devices such as the Hack-RF, USRP, Blade-RF and others have been documented for this use \cite{RN4} \cite{RN9}. These devices are all examples of Software
defined radios that are capable of duplex operation. This combined with open source software, as described in \cite{RN16} and \cite{RN57}. The former is a software
package, GPS-SDR, that converts a compatible SDR into a GPS receiver wihtout any knowledge of GPS or signal procssing required. With some understanding this could be
modified to capture the raw signal for use in a Meaconing spoofing attack. The later refers to the setup and use of the GPS-SDR-SIM program. This program provides a
method for producing a modulated GPS signal for transmission. It requires the RINEX file for the date and time of the spoof attack as well as a set of coordinates to
reproduce. Since it is open source it can be easily modified for any use case. 

Software based GPS simulation tools similar to that of GPS-SDR-SIM have also been created using a combination of C and MATLAB \cite{RN15}. In this example the author used
C to ensure processing efficiency of the modulated signal, and used MATLAB to simulate multipath errors and uncertainty. The resultant file was then saved to the hard
drive of the host computer. This could then be loaded into an SDR for transmission, although this was not specifically touched on in the paper.

Over recent times there has been an increase in the industries that rely on the timing and postiional data provided by GNSS systems. One such industry is autonomous
vehicles, in particular drones. Drones can be used for hobbies, professional photography/ videography or for surveilance purposes. They are small and can be controlled from great
distances. All commercially available drones have some form of GNSS/GPS location service built in, and as such they become a target for spoofing attacks. Some drone
manufactueres have built in auto landing features in the event that the drone flys into restricted airspaces. This was to combat civilians who were flying within
airports, causing safety and security issues. This is one such way that spoofing attacks could be utelised now and in the future, by transmitting a signal that would make
the drone perceive itself to be in a restricted airspace and land. This kind of attack has been successfully conducted as shown in \cite{RN4}. 

There has been research into ways that GPS spoofing attacks could be used in road navigation scenarios. \citeauthor{RN9} were able to implement an algorithm for road
network modelling and navigation spoofing using GPS. This algorithm coupled with a HackRF SDR meant that the authors were able to create a lunchbox sized spoofing device.
Although in research it was found that the victim devices were able to register a difference in location from network based sources and GNSS based sources, the victims
prioritised the location resolved from GNSS sources over network or cellular.
While this attack strategy may be successful against people not familiar with their surrounding area, someone who is familiar or is paying close attention should be able
to tell they are being led to an incorrect area. Where this is less likey to be the case is with driverless vechils.
In the paper written by \citeauthor{RN25} \cite{RN25} regarding driverless vehicle safety, it was noted that there is a significant threat to these types
of systems that rely heavily on reliable GPS signals. Although there have been proposed solutions to this problem through the use of 
other sensor information available locally to each vehicle and in the form of an ad-hoc network known as V2V (vehicle to vehicle) and more broadly
V2X (vehicle to everything) \cite{RN17}, this is still in its infancy and will require joint work from all vehicle manufacturers. 

\citeauthor{RN12} commented on the effect of GNSS spoofing of a cooperative victim. That is when someone is willing to aid the attacker
in performing an attack. This may be implemented to circumvent position based restrictions or if being GPS tracked during certain activities.
\citeauthor{RN12} used the example of a fisherman wanting their GNSS receiver to falsely report the boat had stayed out of protected areas \cite{RN12}.

In 2012 \textcite{RN6} investigated the different spoofing and antispoofing techniques available. The author noted that spoofing attacks can be divided into 3 main categories:
GPS signal simulator, Receiver-Based spoofers and Sophisticated Receiver based spoofers. These attack strategies come about because of vulnerabilities in the GPS system.
These vulnerabilities can be described in the three operational layers of GPS, signal processing, data bit, and position/navigation solutions.
Antispoofing can be broken down into 2 groups spoof detection and spoof mitigation, with each of these being able to be further broken into subcategories.
The effectiveness of each spoof detection technique was tabulated and compared. As was the spoof mitigation techniques. Each detection and mitigation
method was given a complexity, effectiveness and spoofing scenario generality rating with notes made about the received capability requirements.
Testing spoofing techniques is difficult to achieve since there are regulations around the emission of EM radiation at certain frequencies and power levels.
There were three methods used to test the spoofing/antispoofing techniques. These were Indoor re-transmissions, spoofing using recorded data (No RF transmission), and
using RF combiners to combine authentication and spoofed signals.
The results that were acquired showed that the commercial GPS receivers were vulnerable to a number of spoofing techniques. It was also
shown that with modest, low complexity spoof detection and mitigation strategies some of the spoof attacks were able to be overcome.
