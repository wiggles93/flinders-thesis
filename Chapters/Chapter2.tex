% Chapter Template

\chapter{Literature Review} % Main chapter title

\label{Chapter2} % Change X to a consecutive number; for referencing this chapter elsewhere, use \ref{ChapterX}

%----------------------------------------------------------------------------------------
%	SECTION 1
%----------------------------------------------------------------------------------------

\section{Introduction}

In this section the previous work in the field will be explored. This will give the reader adequate understanding of where 
advances in the research area can be made.\todo{Write intro to the lit review}

\medskip
This is a list of things that need to be in a lit review:
\begin{itemize}
    \item Introduction
    \begin{itemize}
        \item Why are you writing the review, and why the topic is important
        \item the scope of the review - what aspects of the topic will be discussed
        \item the criteria used for your literature selection (types of sources used and dates etc.)
        \item the organisational structure of the review
    \end{itemize}
    \item Body Paragraphs
    \begin{itemize}
        \item historical background
        \item methodologies
        \item previous studies
        \item mainstream vs. alternative viewpoints
        \item principal questions being asked
        \item general conclusions being drawn
    \end{itemize}
    \item Conclusion 
    \begin{itemize}
        \item Main agreements and disagreements of the literature
        \item any gaps or areas of further research
        \item my overall perspective on the topic
    \end{itemize}
\end{itemize}
\medskip
Checklist for a literature review. Have I:
\begin{itemize}
    \item Outlined the purpose and scope
    \item Identified appropriate and credible (academic/scholarly) literature
    \item Recorded the bibliographical details of the sources
    \item Analysed and critiqued your readings
    \item Identified gaps in the readings
    \item explored methodologies / theories / hypotheses / models?
    \item discussed varying viewpoints
    \item written an intro, body and conclusion
\end{itemize}
\bigskip
Items that I want to cover in the Lit review
\begin{itemize}
    \item Using SDR (USRP) to receive GPS signal
    \item Using SDR to transmit GPS signal
    \item Algorithms for generating spoofed signal
    \item GPS Anti-spoofing techniques
    \item GPS spoof detection techniques
    \begin{itemize}
        \item This will be what will need to be overcome in the implementation of the spoofing transmitter.
    \end{itemize}
    \item OPTIONAL: Difference between Block II and block III satellites
\end{itemize}
\medskip
Previous research into the use of SDR (software defined radio) for GPS spoofing uses has lead to common conclusions.
That is, that the use of open source cannot  software to ease the development. Namely multiple previous attempts at GPS spoofing
used the GNSS-SDR program for reception and the gps-SDR-sim for transmission. Both of these are available from GitHub as free
and open source programs. The issue of GPS spoofing is not uniquely that of position. Since GNSS systems have highly accurate time keeping
facilities, in the form of atomic clocks, they are used to maintain accurate time of critical infrastructure like power networks, telecommunications \todo{list things that rely on GPS timing}\\

Spoofing can be defined as an intentional interfering signal that aims to make GNSS receivers produce incorrect positioning data\cite{RN8}.

\bigskip
From literature the consensus is that the same thing that has made GPS ubiquitous with navigation and 
positioning has also made it a simple target for exploitation and manipulation, that is the workings
of the infrastructure are well known and public and are transparent and predictable \cite{RN7} \cite{RN4}. This is problematic since this infrastructure
is seen as a critical service by many industries including utility management, healthcare/ emergency services and security.
Having such a system so prone to threats is not ideal. The further development of SDR platforms has driven down the cost of launching
such exploits. Devices such as the Hack-RF, USRP, Blade-RF and others have been documented for this use \cite{RN4} \cite{RN9}.

\medskip

An issue that arises when attempting to spoof a mobile device is that these devices are also able to determine location based off of
network connection. \todo{insert citation for finding location from cellular/WiFi connection}
K. Zeng et al. \cite{RN9} found that when these kinds of devices registered a difference in location between network based and GPS based
that they would prioritise the GPS based location. \todo{I think this is wrong}\emph{This is less of an issue than at first glance since the prevalence of VPN (virtual private networks)
are increasing for circumventing geolocking features. Therefore to ensure that GPS location services still function when a VPN is active
the GPS position should be prioritised.} There are other methods of detecting a faulty GPS signal in software and hardware that are
more accurate and allow for more flexibility of the device usage. 

\medskip

There have been documented cases of GPS spoofing attacks in the field \todo{add in reference to Russian gps spoofing attacks}. So while this paper 
will be focusing on the methodology of performing a spoofing attack it will be in the context of furthering understanding to be able to develop 
more robust anti-spoofing or spoofer locating algorithms.

\bigskip

\citeauthor{RN12} \cite{RN12} commented on the effect of GNSS spoofing of a cooperative victim. That is when someone is willing to aid the attacker
in performing an attack. This may be implemented to circumvent position based restrictions or if being GPS tracked during certain activities.
\citeauthor{RN12} used the example of a fisherman wanting their GNSS receiver to falsely report the boat had stayed out of protected areas.
In the paper written by \citeauthor{RN25} \cite{RN25} regarding driverless vehicle safety, it was noted that there is a significant threat to these types
of systems that rely heavily on reliable GPS signals. Although there have been proposed solutions to this problem through the use of 
other sensor information available locally to each vehicle and in the form of an ad-hoc network known as V2V (vehicle to vehicle) and more broadly
V2X (vehicle to everything) \cite{RN17}, this is still in its infancy and will require joint work from all vehicle manufacturers.   

\bigskip

Cryptography has been investigated by multiple sources as a solution to many aspects of the GNSS spoofing problem. For constellations that have been operational for many
years like GLONASS and GPS, this requires alterations to the space segment. Since 2013 Galileo developers have been working on a one to many cryptographic solution
similar to that used to secure internet communications. This has manifested in the OS-NMA (Open Service Navigation Message Authentication). This allows modern high
performance receivers to be able to enable the cryptographic solution, while still maintaining backwards compatibility for less capable receivers or for those that do
not require the extra security. \todo{add references} \\ 
While this thesis will be concentrating on spoofing and countermeasures of the GPS satellite constellation, it is worth noting that this kind of implementation will be
considered as the existing constellations are updated at end of life.

%-----------------------------------
%	SECTION 2
%-----------------------------------
\section{Annotated Bibliography}
%-----------------------------------
%	SUBSECTION 1
%-----------------------------------
\subsection{Spoofing Attack References}

\textbf{\emph{Hijacking Unmanned Aerial Vehicle by Exploiting civil GPS vulnerabilities Using SDR} by X. Zheng et al.}\\
As written by X. Zheng et al. \cite{RN4} there has been an exponential increase in the number of UAV vehicles being used in recent times.
This increase has been driven by the advent of aerial photography of both professionals and amateurs since the cost of entry is so low. There are
also companies looking to implement delivery services using UAV, or drone, technology.
This poses security and safety concerns with such vehicles having such a reliance on GPS for accurate positioning. Having such an open and transparent
infrastructure such as that of GPS makes the signals easy to reproduce for nefarious purposes. Adding to the concern is the increase in 
affordable and high performance SDR platforms as well as open source software packages designed to minimise complexity of performing such tasks. X.Zheng et al.
showed that performing spoofing attacks on UAV vehicles is cheap and simple to control. Zheng performed 3 different hijacking attacks on the target
UAV, namely forcing the drone to land, guiding the drone to an attacker dictated area and forcing the drone to land in an attacker dictated area.
Two of the attack methods were successful at a rate of approximately 80\% and 90\%. The unsuccessful test was attempting to force the drone to land in a specified location by exploiting the 
'go home' safety feature. This required jamming the communication channel between the controller and drone. After experimentation it was shown
that this required advanced hardware and methods to achieve. Zheng also provided some recommendations for improving the security of the GPS system of, in particular
DJI Phantom 3 drones.
Future works would improve the success rate of the attacks that were successful as well as decrease the time for a successful attack. Work could 
also be put into jamming the control channel between the controller and drone.

\medskip

\textbf{\emph{Practical GPS location spoofing attack in Road Navigation} by K. Zeng et al.} \\
K. Zeng et al. provide insight into the security concerns of leaning so heavily on the GPS system in general, and more important to this
paper is the reliance on GPS for navigation. In their paper \cite{RN9} K. Zeng et al. were able to develop a practical algorithm
implementation for road network modelling and navigation spoofing using GPS. The attack model used was a hardware based approach. 
The proposed attack strategy leveraged a lunch box sized portable spoofer with the Hack-RF at the heart of the device. A lot of the effort
of this paper went into the production of the dynamic modelling of road network algorithm. This was done using an XML (extensible markup language)
version of an area of streets as provided by OpenStreetMap. The documentation of this algorithm will become 
useful when moving from a static based spoof attack to a dynamic one.

\medskip

\textbf{\emph{GNSS-SDR: An Open Source tool for Researchers and Developers} by C. Fern\'andes-Prades} \\
The software has undergone continual development since this paper was written. Therefore more recent version of article should be found.
Although information may still be relevant \cite{RN16}.
In 2011 C. Fern\'andes-Prades et al. developed an open source piece of software for the development and implementation of software defined radio based
GNSS receivers. At the time there was a void of accessible software radio solutions that allowed the used to investigate how a GNSS receiver worked.
The author mentions existing commercial products, as they allow for the usage of the data that GNSS systems provide.
The paper walks through the architecture of the GNSS-SDR program.
\medskip

\textbf{\emph{GPS Vulnerability to Spoofing Threats and a Review of Antispoofing Techniques} by A. Jafarnia-Jahromi} \\
In 2012 \textcite{RN6} noted that with the dependency that society has developed on GPS it has become a viable attack target. One type attack is
so called spoofing. Spoofing and anti spoofing algorithms and techniques have become an area of interest for researchers of late. \citeauthor{RN6}
investigated the different spoofing and antispoofing techniques available. The author noted that spoofing attacks can be divided into 3 main categories:
GPS signal simulator, Receiver-Based spoofers and Sophisticated Receiver based spoofers. These attack strategies come about because of vulnerabilities in the GPS system.
These vulnerabilities can be described in the three operational layers of GPS, signal processing, data bit, and position/navigation solutions.
Antispoofing can be broken down into 2 groups spoof detection and spoof mitigation, with each of these being able to be further broken into subcategories.
The effectiveness of each spoof detection technique was tabulated and compared. As was the spoof mitigation techniques. Each detection and mitigation
method was given a complexity, effectiveness and spoofing scenario generality rating with notes made about the received capability requirements.
Testing spoofing techniques is difficult to achieve since there are regulations around the emission of EM radiation at certain frequencies and power levels.
There were three methods used to test the spoofing/antispoofing techniques. These were Indoor re-transmissions, spoofing using recorded data (No RF transmission), and
using RF combiners to combine authentication and spoofed signals.
The results that were acquired showed that the commercial GPS receivers were vulnerable to a number of spoofing techniques. It was alsox
shown that with modest, low complexity spoof detection and mitigation strategies some of the spoof attacks were able to be overcome.

\medskip

\textbf{\emph{\citetitle{RN23}} by \citeauthor{RN23}} \\
In \citeyear{RN23} \citeauthor{RN23} \cite{RN23} showcased the development of a civilian GPS spoofing device as a means of assessing the potential spoofing 
threat at the time. The main outcome of this experiment was to see the effects and help to develop future defences against spoofing attacks. In addition to 
presenting a spoofing device analyses of the GPS spoofing threat assessment and spoofing defence techniques was also provided. 
The design of the spoofer module, which was a software component of the receiver, was described in detail. This information will be of great use for understanding 
DSP and GPS spoofing although it out of the scope of this section. 
The threat assessment of GPS signal spoofing was summarised into three categories, simple, intermediate and sophisticated attacks. A simple attack is that from a
GNSS signal generator, intermediate from a software radio and complex was in the form of coordinated attack by multiple phase locked radios. 
Spoofing defences suggested at the time of publish were as follows; Data bit latency defence and vestigial signal defence.

Sophisticated attacks were found to be effective against commercial GPS units and many anti-spoofing techniques with cryptographic methods being the most 
effective way of circumventing the attacks.

This spoofing device was developed on a DSP platform, which at the time was an appropriate choice. Given the leap forward in power and flexibility
of SDR platforms since this paper was published it should be noted that the use of an SDR for this purpose would be recommended. At the time of publish
it was noted that the cost of including multi band spoofing capabilities to a spoofing device would be prohibitively expensive. This again has been made accessible
by SDR platforms. 

\medskip

\textbf{\emph{\citetitle{RN21}} by \citeauthor{RN21}} \\
In \citeyear{RN21} \citeauthor{RN21} \cite{RN21} developed a system to perform a series of spoof attacks on UAV's based on an SDR \cite{RN23}. The goal of this was to be able to 
control the flight path of the UAV without raising any alarm's from the victim. This paper establishes the required conditions in which a UAV will
be susceptible to being captured from a spoofing attack, as well as the range of post capture control that the spoofer will have over the victim.
From testing it was found that spoof attacks were successful from up to 50m and up to a velocity of 10m/s.
Simulations were produced for analysis of post capture control of the UAV. 
When testing, both covert and overt spoofing methods were used, distinguished by whether or not the spoofer made an attempt to avoid detection. For the most part there
was no practical difference between covert and overt methods since the commercial GPS units did not trigger any anti spoof mechanism when being subjected to an overt
attack. Field testing showed that the spoof attempt caused unrecoverable navigation errors which resulted in the UAV crashing. Future work should increase the
sophistication of the attacks. This may decrease chances of the drone crashing due to unsuccessful attacks.

\medskip

\textbf{\emph{\citetitle{RN28}} by \citeauthor{RN28}} \\
In \citeyear{RN28} \citeauthor{RN28} \cite{RN28} used a low cost SDR device to perform a GPS spoofing attack. This method can be used to alter the location or time
of the victims. Other positioning methods and techniques to spoof them were also examined. This was performed such that advice on how to prevent such attacks could be 
presented. As mentioned in other sources, GNSS systems are tightly entwined with the day to day operations of people, commercial business and government entities alike.
It was noted that the signals presented to the receivers are usually trusted without any authentication or other checking. The authors show that this trust can be exploited
without physical access to or altering the software of the target device. One of the signal properties that makes spoofing simple is the received signal power from space
which is less than -130dBm making overpower the legitimate signal a simple task. 
To create a spoofing device an SDR platform was used (Hack-RF) in conjunction with the open source program GPS-SIM-SDR. This program is able to generate a GPS signal
at a chosen coordinate when given the correct BRDC (broadcast ephemeris data) in the form of RINEX files. A simple use of this software is to generate a static location
but it is achievable to create a dynamic signal. Hack-RF was chosen as the SDR platform because of its open source nature, it also had the required specifications to
perform the spoof attack. The spoofing of WiFi signals was deemed to be out of the scope of this thesis and not considered.
It was shown in this paper that the use of an SDR and open source software was able to fool client devices such as apple smartphones into thinking they were else where.
This included specific apps that require location data such as Uber and DiDi. The authors provided a list of recommendations to minimise the risk of being spoofed. 
These recommendations are non trivial and some required alterations to how the receiver processes the signal data. These methods have been considered in greater
detail in other references.

\medskip

\textbf{\emph{\citetitle{RN30}} by \citeauthor{RN30}} \\
In \citeyear{RN30} \citeauthor{RN30} \cite{RN30} investigated the requirements to performing a successful GPS spoofing attack on individuals and groups of receivers
using civilian or military signals. A civilian GPS signal generator was used to create the spoofed signal for testing purposes.
The authors showed through testing that any number of receivers can be easily spoofed to a single arbitrary static location, whereas
when preserving the their constellation there are only a few transmission locations that were able to be used \todo{Reword this, read the paper to understand}.
From the experiments it was found that there are 4 parameters that have required values in order to successfully spoof a GPS signal. That is the relative signal 
power must be $\geq+2dB$, the constant time offset must be $\leq 75ns$, location offset must be $\leq 500m$ and the relative time offset must be $\leq 80ns$.


\medskip

%-----------------------------------
%	SUBSECTION 2
%-----------------------------------
\subsection{Anti-Spoofing References}
%-----------------------------------
%	SUBsubSECTION 1
%-----------------------------------
\subsubsection{Spoofing Detection}

\textbf{\emph{GNSS Spoofing and Detection} by M. Psiaki et al.} \\
\textbf{This is a really good reference for the state of play in the spoof attack and defence landscape. Spend time reading about the techniques and 
gain an understanding} \todo{Add comparison to \cite{RN11}}.\\
In 2016 M. Psiaki et al. \cite{RN12} reviewed the state of GNSS spoofing and spoofing defences. Since 2008 there has been an ever increasing 
interest in GNSS spoofing. A spoofer could potentially target both military and civilian infrastructure including aircraft or marine vessels which
rely on GNSS signals especially in low visibility conditions. Civilian infrastructure like cell phone towers, power grid monitoring and stock markets
use GPS signals for precise timing. This paper looks at the details of the current methods of implementing a GPS spoof attack as well as the details
of known anti spoofing techniques. Attack methods tested were self consistent spoofer, Meaconing and estimate and replay attacks, other advanced techniques.
Defence techniques can be broken down into three categories; pseudo-range RAIM (receiver autonomous integrity monitoring),
looking at the difference between a true and spoofed signal, and looking for interaction between true and spoofed signals. Pseudo-range RAIM is considered
too weak against modern spoofing attacks and was not considered in this publication. The specific methods of defence tested in this paper  
were advanced signal processing techniques, encryption based, drift monitoring, signal geometry and multi pronged spoofing defence.
A \todo{include table in lit review}table was generated to visualise which defence strategies were effective against which attack strategies. 
It was found that all attack strategies that were used in the report were effective against COTS (commercial off the shelf) GNSS receivers. The authors
recommend COTS GNSS receiver producers implement, at a minimum, a rudimentary defence strategy against spoof attacks. 

\medskip

\textbf{\emph{\citetitle{RN32} by \citeauthor{RN32}}} \\
\textbf{Use this as basis for lit review and sprinkle other sources in when they are able to add to what the authors are saying here} \\
\textbf{This is a really good resource for how GNSS systems work. Use this extensively in the introduction}
In \citeyear{RN32} \citeauthor{RN32} \cite{RN32} provided a review and analysis of the GNSS spoofing threat as well as the countermeasures of the time. This report aimed
to fill the gap in 3 main areas of GNSS spoofing research, namely assessing the exact threat scenarios for commonly cited targets, investigate the practical impediments
of performing a spoofing attack in the field and lastly to survey and asses the performance of proposed GNSS spoofing defences. When defining spoofing, \citeauthor{RN32}
determined that a power ratio (spoofed vs authentic) of just 1.1 was sufficient to force the receiver to move onto the spoofed signal. And with authentic GNSS signal
strengths of approximately $-153dBW$ to $-160dbW$, this ratio is easy to achieve. \todo{how many watts would be required to get the ratio 1.1?} It was noted that the
only detectable differences between a legitimate and spoofed one is the timing difference, signal angle of arrival, signal strength, Doppler shift and SNR. Most receivers
as of \citeyear{RN32} were not capable of registering these differences. The AGC (Automatic Gain Control) functionality common on GNSS receivers adjusts the signal gain
to compensate for any changes in the received signal strength, and in doing so makes the receiver more susceptible to a spoofing attack. The two main spoofing attack
types analysed were \emph{Meaconing} and \emph{SCER}. SCER is an evolution of the Meaconing attack where a single satellite signal is rebroadcast after a set delay.
Meaconing is the simplest spoofing attack and is a capture and re-transmission attack. This attack is difficult to perform on the military P(Y) code, but simple to perform
for the civilian C/A code. The most commonly referred to GNSS spoof attack targets came in the form of Power Disruption networks, shipping, aircraft, trains, criminal
security tags and mobile phones. Each of these examples were analysed in detail for potential attack vectors.
It was also noted that there are significant practical difficulties to performing a spoofing attack outside of an academic setting where variables can be tightly
controlled. First and foremost the cost of the hardware, while declining quickly, is still out of reach of many of the general public costing upwards of \$2300 at the
time. Current pricing for the USRPN210 in Australia is approximately \$4000. 
Those without the technical know how could purchase an off the shelf GNSS simulator unit, however these start at \$20,000 for a multi constellation generator. Practical
difficulties continue with distance from the target, which, due to the inverse square law of radio propagation means to ensure a sufficiently high SNR that distance from
attacker to target should be minimised. This is made difficult if the target is moving. Other difficulties were raised including need for direct line of sight and the need for
multiple simulators for multi GNSS receivers. 
While it was found that current generation GNSS receivers were susceptible to spoof attacks, there were countermeasures that could be employed to either detect or
mitigate the spoofing threat. Of the proposed GNSS spoofing countermeasures all can be summarised with four categories, Signal processing defences, Cryptographic
defences, Correlation with other timing sources, Radio spectrum and antenna defences. It was found that at least 1 method from each of the categories was highly effective
at defending against an attack. Given the practical difficulties in performing a spoofing attack and the effective defences that have been developed (although not used in
the wider community as yet) it was concluded that the threat posed by GNSS spoofing was low but with the potential to increase as the practical implementation became
easier and cheaper and if companies did not implement countermeasures.
Recommended future work was based around real time detection and location of spoofing attacks.

\medskip

\textbf{\emph{Spotr: GPS Spoofing detection via Device Fingerprinting} by M. Foruhandeh et al.}\\
As discussed in \cite{RN7} GPS (Global Positioning System) has become the de facto positioning infrastructure for civilian and military usage and while there are other
satellite based positioning systems that are becoming more popular (GLONASS, Galileo, BeiDou) GPS is still the one that most people think of when thinking about GNSS. GPS
provides this service globally and efficiently although through its ubiquity security concerns have become apparent. Attacks have been carried out in both an academic and
industry setting. The authors believed that strengthening of the GPS framework against manipulation and exploitation must be a priority. Proposed methods of hardening GPS
have been previously proposed, however these were typically abandoned due to expense, complexity, cost or lack of robustness. A method of creating unique satellite
"fingerprints" was developed and tested against all known spoofing attacks at time of publish. This fingerprint algorithm was able to determine if a transmission was
authentic or spoofed. This fingerprinting algorithm implementation was found to be fast, simple, cost effective and robust when compared to previous hardening attempts.

\medskip

\textbf{\emph{\citetitle{RN10} by \citeauthor{RN10}}} \\
In \citeyear{RN10} \citeauthor{RN10} \cite{RN10} proposed a method for detecting the presence of GPS spoofing signals through the use of two or more receivers. Previous
literature had investigated the use of spoof detection using single receivers. These techniques were varied and ranged from simple methods like monitoring the power level
of the received signal, to more complex like analysing the presence of vestigial peaks in the correlator output. There have also been methods proposed that compare
signals with that of non-GPS signals for example an IMU. The multi receiver design that \citeauthor{RN10} proposed works on the principal that the received signal from a
spoofing source will have identical characteristics (relative time of arrival) at each of the receivers. When there is no spoofer present there will be unique signal
characteristics at each of the receivers. The concept of detecting a spoofing event was to compare the position of each receiver against the known relative positions. This
was chosen as it will not require any hardware modifications and the software processing can be done by an external processor, making it suitable for retrofitting to
existing systems. Four simulations were tested from a Neyman-Pearson perspective (1) two receivers with known
absolute locations, (2) two receivers with known separation and orientation but unknown absolute location, (3) two receivers with known separation and unknown orientation
and absolute location and finally (4) three receiver example. The testing was performed under the assumption that the spoofing transmitter only had one antenna. In the
first case there was approximately 1\% chance of a false alarm while the probability of detection was found to be intractable.
It was noted that to enable an inital statistical analysis of the solution, some simplifying assumptions were chosen. As such, future works should attempt to reduce the
simplifications where possible to create results that are applicable to real world scenarios.

\medskip

\textbf{\emph{\citetitle{RN1}} by \citeauthor{RN1}} \\
In \citeyear{RN1} \citeauthor{RN1} \cite{RN1} proposed a method of GPS spoof detection that would allow for the localisation
of the source to be determined. Specifically this method is designed to work with vehicular GPS systems and uses existing 
V2V (vehicle to vehicle) communications systems. Using a network of cooperative vehicles the spoofer could be localised by analysing the respective
Doppler shifts. After laboratory testing on stationary spoofers, it was found to be effective at providing an approximate location at up to 200m away.
When the spoofer was moving, this was decreased to 50m and was movement direction agnostic. To confirm these results an experimental setup would need
to be created and experiments run. It would be expected that the real world results would not be as favourable as the simulated ones, since a number of
assumptions were made that simplified the simulations. For example the spoofer was moving with a constant velocity in a perfectly straight line.

\medskip

\textbf{\emph{\citetitle{RN24}} by \citeauthor{RN24}} \\
In \citeyear{RN24} \citeauthor{RN24} \cite{RN24} \footnote{https://www.youtube.com/watch?v=tsrOKeIelLc} proposed a method for detecting and localising
a GPS spoofing attack called Crowd-GPS-sec. This system was developed to specifically detect and localise GPS attacks directed at UAV (drones) and commercial
airliners. The authors make note that unlike other attempts to increase the security of GPS receivers, and GPS systems in general, no modifications are required
for the receivers or GPS satellites. Alternatively Crowd-GPS-sec utilises crowd sourcing to monitor the air traffic. The aircraft broadcast their GPS derived position
periodically for air traffic control purposes. By leveraging crowd sourcing via the use of ADS-B broadcasts these aircraft are constantly tracked. 
By analysing the difference in position for each aircraft spoofing presence can be detected. Simulations from the opensky network indicate that spoof detection can be 
completed within 2 seconds. Spoofer localisation can also be achieved with relatively high accuracy of 150m from 15 minutes of monitoring.

\medskip

\textbf{\emph{\citetitle{RN31}} by \citeauthor{RN31}} \\
In \citeyear{RN31} \citeauthor{RN31} \cite{RN31} investigated the use of smartphones as a means of GPS spoof and jamming detection, by evaluating built in
sensors for their suitability. As mentioned in other sources GPS security is an important area of research since there are so many areas of everyday life that depend on reliable GNSS signals.
The author noted that in 2016 the Android API included access to the Raw GNSS properties including navigation messages, pseudo-ranges,
Doppler frequencies, constellation status and others. It's this reason that Android was chosen for this experimentation. From these properties a GPS spoof detection program
can be developed without the need to modify the hardware of the GPS receiver or the transmitting satellite. 
From the available information and sensors of the chosen phone, 4 sensor combinations were chosen 1) network location provider, 2) AGC (automation gain control) and C/No engine,
3) inertial sensor data and 4) pseudo-range residual metrics. It was noted that there exists an app that combines all of the mentioned methods together "GNSSAlarm" which was
used for testing. From testing it was found that phones tend to give a higher priority to GPS based location than that of other methods (network based). 
From testing it can be seen that each of the metrics available can be used to combat certain types of spoofing attacks, and when used together as in GNSSAlarm, 
a robust solution is available.

\medskip

\textbf{\emph{\citetitle{RN42}} by \citeauthor{RN42}} \\
In \citeyear{RN42} \citeauthor{RN42} \cite{RN42} proposed a method of GPS signal spoofing detection using pre-existing cellular network infrastructure. Previously
proposed methods for GPS signal detection have typically required special hardware and are difficult to deploy since they are based on the physical properties of the
received GPS signal. This new proposal uses broadcasts from cellular towers to validate the position recevied by the GPS signal. Parameters from the cell towers used to
determine the validity of the GPS signal include the number of in-range base stations, the distance to those stations and the received power of the signal. This method
was tested against a real spoofing attack and it was found that the there was 0\% false positive complete detection of the spoofing signals tested with negligable delay.


%-----------------------------------
%	SUBsubSECTION 2
%-----------------------------------
\subsubsection{Spoofing Mitigation}

\textbf{\emph{\citetitle{RN34}} by \citeauthor{RN34}} \\
In \citeyear{RN34} \citeauthor{RN34} \cite{RN34} provided a survey of the current GNSS spoofing and jamming mitigation methods current to that point inn time. More
emphasis was put into the detection and mitigation of intentional interference such as spoofing and jamming of GNSS signals in aviation. This meant there were extra
constraints when considering appropriate countermeasures. Keeping antenna numbers and placements were of chief concern. The paper summarised the history of navigation
solutions used in the aviation industry dating back to 1947. From 1973 on wards GPS has been in use, and only expanded to include other GNSS constellations. Other types
of navigation solutions are out of the scope of this document.
\citeauthor{RN34} separated the interference's into man-made and channel-based, with the man-made being further separated into intentional and unintentional. Examples of
intentional interference were jamming, spoofing and meaconing, with unintentional interference coming from adjacent channels and co-channels. Factors like atmospheric
conditions or space weather, multi-path communications, Doppler effects and scattering all form unintentional interference. A detailed summary of mitigation strategies
based on the attack/interference type was provided. By grouping all of the resources together this paper was able to produce a single repository where defences against
all attack categories mentioned above. Jamming and spoofing signal models were presented, with a class baseband signal model and a list of unknown parameters. Jammers
were classified into five different classes based on the complexity of the attack. Single/multi tone AM/FM type attacks made up class 1 jammers, single and multi chirp
attacks made up class 2 and 3. Chirp signals with frequency bursts were class 4 jammers and class 5 were DME-like/pulse jammer. Spoofers were divided into simplistic
spoofer, intermediate spoofer, sophisticated spoofer and meaconer. The unintentional interference models were also discussed and it was noted that adjacent channel
interference is typically harmonics leaking over from other frequency bands. This may be from DVB or LTE sources and is modelled as AM or FM tones. Co-channel
interference is most often caused by other GNSS services transmitting at the same frequency. This is modelled as increased AWGN (Additive White Gaussian Noise). A summary
of the different detection methods was produced. Detection of a single interference type is simple and can be determined using PDFs (probability density functions). The
scenario in which there is no interference will be Gaussian, whereas in the presence of interference the received signal may not be depending on the interference type.
Front end detection uses the raw signal before going though the ADC, commonly AGC detectors are used. Since the AGC will attempt to maintain a steady input power it can
be monitored to see if the gain requirement suddenly drops, indicating a high power signal and most likely a spoofer. The next type of detection method is the
pre-correlator type \todo{Finish}

\medskip

\textbf{\emph{\citetitle{RN36}} by \citeauthor{RN36}} \\
In \citeyear{RN36} \citeauthor{RN36} \cite{RN36} proposed an implementation of the Galileo OS-NMA for embedded devices running ARM based processors. The OS-NMA is an
attempt to include spoofing countermeasures within the message structure from the satellite. Software profiling was performed with performance comparisons undertaken
between different platforms. There are OSNMA ready commercial hardware receivers, however the authors decided that a software implementation would be best suited as new
features are easily added and the algorithm can be easily accessed and modified as needed. An overview of the OS-NMA was provided with parameters, values and
descriptions. Testing was performed on a Dell PC with Xeon processor and an ODroid running ARM based exynos processor. It was shown that the ARM based device suffered
with execution times ballooning. However the receiver was able to decode the OS-NMA code susseffully. Future work will go into optimising the software for different
platforms and utilising more powerful hardware to close to gap with more powerful PC machines.

\medskip

\textbf{\emph{\citetitle{RN37}} by \citeauthor{RN37}} \\
In \citeyear{RN37} \citeauthor{RN37} \cite{RN37} produced a report on the addition of an authentication system within the Galileo GNSS constellation as means of
circumventing interference of the navigation signal. This interference may be intentional in the case of a jamming or spoofing attack of unintentional in the case of
irregular or intense atmospheric weather conditions. This architecture would be added to it's open service and be named the NMA (Navigation Message Authentication). The
NMA cryptographic data is mostly unpredictable, which can be used to protect against replay attacks like meaconing attacks. The unpredictability code was chosen to be
stored in the "Reserved 1" slot of the I/NAV page. This section is 40bits wide. The unpredictability is determined by how the data is packaged with the I/NAV message and
how the bits are coded into symbols and interleaved. By storing the first chip of this symbol a reciver can  create a synthtic sequence whose correlation will be low if
the signal has been subject to interference. The authors determined that the inclusion of the NMA code reduced the effectiveness of replay based spoofing attacks.
Care must be taken if trying to spoof a victim that has access to either Galileo GNSS or multi-constellation GNSS since this functionality may be preasent and active. 

\medskip

\textbf{\emph{A Low complexity GPS anti-spoofing method using multi antenna array} by S. Daneshmand et al.} \\
In 2012 S. Daneshmand et al.\cite{RN8} proposed a low computational complexity anti spoofing module for use with GPS receivers.
It was mentioned that previous anti-spoofing research was done in two categories, spoofing detection and spoofing mitigation.
Most of the effort has been put into spoofing detection algorithms and less on mitigation techniques. 
Previous proposed spoofing mitigation devices used the fact that GPS spoofing devices typically transmit multiple PRN's. 
Therefore comparisons can be made between the spoofed and authentic signals. Although this is computationally complex. This proposed
device uses a multi antenna array to find the signal with most power and beam form in such a way that the antenna null is pointing in
that direction. This relies on the assumption that spoofed signals will have a measurably higher spatial energy than that of the authentic signal
as well as the authentic signal propagating from an area that is not also covered by an antenna null. Although this cannot be guaranteed.
To maximise the performance of the module, the authors also propose a technique to amplify the authentic signals as well as reduce the noise floor level.
S. Daneshmand mentioned that with some minor modifications the proposed module would be able to operate with GLONASS and Galileo systems as well. 

\medskip

\textbf{\emph{Practical Cryptographic Civil GPS signal authentication} by K. Wesson et al.} \\
K. Wesson et al. \cite{RN13} proposed a method for counteracting GPS spoofing on the popular GPS civil band. The proposed method would use cryptographic
authentication based techniques to ensure that signals were not fake and did originate from an authorised source.
\citeauthor{RN13} provided a detailed overview of the entire cryptographic authentication system.
The method employed by \citeauthor{RN13} included adding digital signatures within the GPS civil navigation message. This type of cryptographic defence
will secure the receiver against replay spoof attacks. Authentication of each civil GPS signal is performed every 5 minutes.
For the proposed authentication strategy a $P_D$ of >0.97 was achieved over the range of $37$-$51 dB-Hz$.
The implementation of such a system would require modifications to the GPS satellite system themselves, as well as a redesign of GPS receivers.
The modifications required for the GPS receivers were documented in detail in \cite{RN13}. For systems that can not easily or readily upgraded to
take advantage of these improvements an external device called "The GPS Assimilator", as proposed in \cite{RN19}, could be used to clean the incoming
signals.
Although it was noted that receivers based on SDR platforms would be able to implement the authentication algorithm easily.

\medskip

\textbf{\emph{\citetitle{RN17}} by \citeauthor{RN17}} \\
In \citeyear{RN17} \citeauthor{RN17} \cite{RN17} noted that the GPS navigation system is susceptible to electro magnetic interference. Sources of interference 
can either be natural or artificial including malicious sources (spoofing). This has the potential to threaten the reliability of the system which has become 
vital in accurate position and timekeeping purposes. One such use is within modern vehicles for positioning, navigation and automation. 
CACC (Cooperative Adaptive Cruise Control) systems are an evolution of ACC (Active cruise control) and include a communication stack for V2V (vehicle to vehicle) communication. 
Vehicles that have a CACC system thus have a number of other sensors which are able to provide situational awareness. These include the for mentioned radio
communications equipment as well as Lidar, radar, camera  IMU's (inertial measurement unit) and access to CAN data from the vehicle systems.
\citeauthor{RN17} devised a method of maintaining awareness of position from a combination of GPS signal as well as the other sensors as part of
the various other systems integrated into the vehicle and from other vehicles. It was shown experimentally that the COTS receivers were susceptible to
to spoofing attacks. It was also shown that the proposed method of spoof detection and removal was effective. Further work will need to be done
to test the defence method on more spoof attack methods. To ensure that this design becomes viable for the commercial world it must also work in real time.

\medskip

\textbf{\emph{\citetitle{RN18}} by \citeauthor{RN18}} \\
In \citetitle{RN18} \citeauthor{RN18} \cite{RN18} pointed out that there is a cycle of new spoofing attack strategies and then counter-measures
against them. A sudden outage to GNSS systems would have a large impact on society. \citeauthor{RN18} provided a list of GNSS-reliant ecosystems including
autonomous transportation, cellular networks and power systems. Disruptions to especially the later two would have a measurable impact. Therefore, it is
important to ensure that receivers of GNSS signals are resilient to intentional and unintentional interference.
Typical COTS GNSS receivers do not have a way of being upgraded since their hardware is set and not able to change. Having a receiver
with upgradable hardware, such as that found in SDR platforms, would make for an ideal GPS receiver. As such SDR's are well suited as GNSS receivers,
however, they are challenged by processing the GNSS data and anti spoofing method. \citeauthor{RN18} performed experiments to see how effective SDR 
platforms were at implementing a mitigation strategy in real time. A reduced complexity MMSE (minimum mean squared error) algorithm was developed for
use on the in house developed SDR platform. This algorithm was compared against MF (matched filter) techniques. It was found that at an ISR (interference to signal ratio)
of 30dB that MF correlators were unable to receive any usable signal, with BER (bit error rate) of approximately 50\%. Using the proposed maximise
correlator, at a 30dB ISR, showed a BER gain of approximately $10^5$. 
However, in testing it was found that most COTS receivers were not able to perform these duties in real time due to the increase in complexity of the antispoofing algorithms.

\medskip

\textbf{\emph{\citetitle{RN11}} by \citeauthor{RN11}} \\
In \citeyear{RN11} \citeauthor{RN11} \cite{RN11} provided an analyses of the vulnerabilities of the GPS signal to spoofing attacks including the
satellite signal itself. \citeauthor{RN11} also proposed anti-spoofing algorithms and provided simulations to showcase the effectiveness of these
algorithms. In general only anti-spoofing methods that could be processed in software without modification to the receiver itself were considered.
In summary there were 10 anti-spoofing methods discussed in the paper; \todo{include summary table} 
\begin{enumerate}
    \item Monitor absolute power of each carrier: \\
    Since the expected reception power of a carrier is approximately $-155 dBm$. Using this a reasonable maximum power
    can be set.
    \item Monitor signal change in power: \\
    Since transmitted power satisfies equation \ref{eq:Power_tx} it can be seen that the received power will be altered significantly by a change in
    distance to the source. Therefore, a sudden change in received power could be an indication of attempted spoofing. However since the received power
    is dependant on other variables like the environment and path that the signal takes, this method is ideally used for static observations.
    \begin{equation} \label{eq:Power_tx}
        P_{tx}=P_{rx}r^2
    \end{equation}
    \item Monitor the relative power: \\
    This method uses the power ratios of different frequency bands to determine spoof activity. As an example there is a $~3dB$ difference between
    the L1 and L2 band signals typically. If the observed power ratio is too far outside of this range then it can be assumed that there is a spoofing
    attack underway. It should be mentioned that ionosphere refraction may also affect the power ratio and should be taken into account.
    \item Bound and compare range rates: \\ 
    Since a moving receiver's phase movement is out of the spoofers control the code and phase range rates can be compared to detect spoofing. 
    Bounding these rates can add to the ability to detect spoofing activity.
    \item Doppler shift check: \\
    It is impossible for the spoofing attack to completely replicate all of the Doppler shifts for all satellites using a single transmitting antenna.
    Therefore, it would be necessary to have as many antenna as there are satellites which make it prohibitive from a spoofing standpoint. This can be
    used to test for spoofing activity.
    \item Cross correlation of L1 and L2: \\
    Based on the way that the GPS frequency bands operate, a cross correlation of L1 and L2 bands should result in a single peak, and due to the 
    difference in signal speed the sign of the cross correlation is known. This method will be unsuccessful if the spoofed is spoofing both carriers.
    \item Residual Analysis: \\
    Under spoofing conditions the received signal will be a combination of the authentic signal, the spoofed signal and random noise. The spoofed signal
    can be removed and if the ratio between the spoofed signal and authentic signal is low enough then the authentic signal can be recovered, otherwise
    the authentic signal is lost. 
    \item L1-L2 range differences
    \item Verify received ephemeris data: \\
    By comparing the received ephemeris data to known ephemeris and almanac data the spoofer will be unable to use its own position as stand in. 
    \item Jump detection : \\
    This method is based around tracking all observables and if there is a "Jump" in a specific observable this could indicate a spoofing attack.
\end{enumerate} 
Many of these methods must co-exist and can be used together to create a more robust anti-spoofing solution. For example if the absolute power is 
not limited, then the residual test will not be of use. It should be noted that while the presence of a spoofer can be determined, and hence 
throw away the related positional data, there has not been any method provided to re-determine the correct position. Therefore, in the case that
spoofing is detected, the attack will become a jamming attack.

Given that this paper was released in \citeyear{RN11} some of this information is outdated. For example it was concluded that GPS spoofing was not formidable
since it could be easily detected. With the advent and advancement of SDR platforms
GPS spoofing hardware has become highly performant and affordable as well, with practical examples of spoofing attacks being \todo{insert reference} documented. 
However this paper provides a very good basis for building more complicated spoofing and anti-spoofing techniques since the underlying GPS system is
relatively unchanged. 

\medskip

\textbf{\emph{\citetitle{RN19}} by \citeauthor{RN19}} \\
In \citeyear{RN19} \citeauthor{RN19} \cite{RN19} devised a conceptual method for retrofitting anti-spoofing techniques to GPS devices without the need to alter the
software or hardware of those devices. This device is called a GPS assimilator and can also be used for increasing the accuracy and robustness of the PVT in weak-signal
environments. From a high level the assimilator will contain an RF front end with multi system receiver and navigation and timing fusion module,  
anti-spoofing modules and an embedded GPS signal simulator. Conceptually this device will be placed between the RF antenna and the GPS receiver. The assimilator thus
performs all necessary signal processing to output an L1 C/A signal that the target GPS receiver will be able to interpret with accurate PVT data. The assimilator can be
used with many location systems including multiple GNSS constellations and cellular connections. Three use cases were included in the paper. These are to protect the time
reference receiver from an unsophisticated spoofing attack, reducing the ionospheric error in single frequency target receivers and weak-signal tracking via CDMA cellular
signal aiding. It was found that in each of these use cases the assimilator was able to improve
the ability of the target receiver to provide accurate PVT data.

By having a separate inline component, especially one based off of SDR technologies, upgrades can be applied to any GPS receiver without needing to make modifications to
the existing system. This reduces cost and complexity of upgrading now and in the future as GPS anti-spoofing technology changes. By using an assimilator a simple single
frequency single constellation receiver can be upgraded to be a multi-frequency, multi-constellation receiver.

Future work on the assimilator include miniaturising the design in order to fit inside of an antenna such that a user would be able to include all of the benefits of using
the assimilator by only needing to change the antenna used. 

\medskip

\textbf{\emph{\citetitle{RN20}} by \citeauthor{RN20}} \\
In \citedate{RN20} \citeauthor{RN20} \cite{RN20} addressed the issue of GNSS signal acquisition when exposed to interference by using an array of antennae.
The approach was a statistical one that leveraged NP (Neymann-Pearson) and GLRT(generalised likelihood ratio test) frameworks to obtain a new GNSS detector 
which was able to mitigate temporarily uncorrelated point source interferences. \todo{get help understanding this paper}

%-----------------------------------
%	SUBsubSECTION 3
%-----------------------------------

\subsubsection{Other References}

\textbf{\emph{Design of complete Software GPS Simulator with low complexity and precise Multi-path Channel model} by A. Elango et al.} \\ 
In 2016 A. Elango proposed a completely software based approach to simulating GPS signals \cite{RN15}. A summary of the difference between analog
and digital based simulation systems is provided, with justification for producing a software based solution. In this proposal the GPS signals are created
using a custom piece of software. The output of this software is a message that has already been modulated, which is saved into a file on the local
storage device. This was then fed into a MATLAB simulation environment for testing. The pseudo code for the development of the program is include in
the publication, and provides a good resource for breaking down the message structure for GPS.

\medskip

\textbf{\emph{\citetitle{RN25}} by \citeauthor{RN25}} \\
In \citeyear{RN25} \citeauthor{RN25} \cite{RN25} produced an overview of current and anticipated future challenges to be faced by the automotive industry in regards to
driverless vehicle security. Driverless vehicles have a large number of sensors and communication infrastructures to facilitate the ability to autonomously drive around
the streets. This increases the attack surface since there are a plethora of different ways to hack or spoof the data to or from the vehicle. Examples of attacks are taking
control of the vehicles and forcing it to stop, or controlling the vehicle to perform an attack by driving into buildings or groups of people. The bulk of this paper
focuses on the many parts that make up the vision and communication system of an autonomous vehicle, however most of this is out of the scope of this thesis. GPS spoofing
techniques were investigated as well as potential countermeasures. Aside from the usual techniques for mitigation and detection, the use of other onboard sensors, namely
radar sensors, could help in the detection process. Future research will continue to focus on hardening the defences of communication based protocols to ensure the data
being shared is reliable, but also as GPS spoofing becomes more accessible from a cost and technical perspective, more effort will need to be put into detection and
mitigation strategies. Since autonomous vehicles must process so much data quickly and efficiently a software based solution would be acceptable since there would be
minimal overhead added, and this means that off the shelf GPS receivers would still be usable. 

%-----------------------------------
%	SUBSECTION 3
%-----------------------------------

\subsection{Literature Review methodology}
\begin{itemize}
    \item Exclusion criteria \\ To help narrow the scope of the literature review certain criteria was selected to be omitted
    \begin{itemize}
        \item Age of source: > 2010 for SDR specific | > 2008 for all others \\ When it comes to the implementation of spoofers using SDR platforms, the most up to date information is required since the software typically used in these cases are open source and prone to frequent changes.\\ This is less of an issue when detailing the operation of the GPS navigation system, or, to a lesser extent, effective spoofing attacks and defences. This is because most off the shelf receivers have not got any spoof mitigation built into their designs \cite{RN12}.
    \end{itemize}
    \item Inclusion criteria
    \begin{itemize}
        \item Application specifics \\ While not all sources will have an application, whether to attack or defend, it should be noted that there was an importance placed on getting sources that applied the theory described.
        \item SDR implementation \\ Since this project relied on the use of an SDR platform for the application of a spoofer it was important that a large number of sources also use a similar platform. This helped in forming a strategy to complete the project. It was also important so that time was not wasted reinventing the wheel.
    \end{itemize}
    \item Keywords
    \begin{itemize}
        \item GPS
        \item GPS Spoofing
        \item GNSS Spoofing defence
        \item GPS UAV Attack
        \item GPS Spoof mitigation
        \item GPS Spoofing autonomous vehicles
        \item meaconing
        \item GNSS threat analysis
        \item GNSS threat survey
    \end{itemize}
\end{itemize}

\subsubsection{Finding sources}
An effective method used to find appropriate sources was to find a recent paper that may not be appropriate but still relevant and to carefully look through
the reference list. By applying the list of exclusion and inclusions as above suitable sources could be found.


%----------------------------------------------------------------------------------------
%	SECTION 2
%----------------------------------------------------------------------------------------

\section{GPS Threat Analysis}
\todo{This might be better suited to being the first Section after the Introduction}
As shown in Table \ref{tab:AttackSum}

\medskip

\begin{table}
    \begin{center}
        \caption{Summary of attack and defence strategies}
        \label{tab:AttackSum}
        \begin{tabular}{ |m{1cm}|m{3cm}|m{5cm}|m{3cm}| }
            \hline
            \textbf{No.} & \textbf{Author} & \textbf{Paper Title} & \textbf{Attack, Detection or Mitigation} \\
            \hline
            1 & \citeauthor{RN7} & \citetitle{RN7} & Mitigation \\
            \hline
            2 & \citeauthor{RN4} & \citetitle{RN4} & Attack \\
            \hline
        \end{tabular}
    \end{center}
\end{table}
