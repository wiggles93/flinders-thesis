% Chapter Template

\chapter{Literature Review} % Main chapter title

\label{Chapter2} % Change X to a consecutive number; for referencing this chapter elsewhere, use \ref{ChapterX}

%----------------------------------------------------------------------------------------
%	SECTION 1
%----------------------------------------------------------------------------------------

\section{Introduction}

In this section the previous work in the field will be explored. This will give the reader adequte understanding of where 
advances in the research area can be made.\todo{Write intro to the lit review}

\medskip
This is a list of things that need to be in a lit review:
\begin{itemize}
    \item Introduction
    \begin{itemize}
        \item Why are you writing the review, and why the topic is important
        \item the scope of the review - what aspects of the topic will be discussed
        \item the criteria used for your literature selection (types of sources used and dates etc.)
        \item the orgonisational structure of the review
    \end{itemize}
    \item Body Paragraphs
    \begin{itemize}
        \item historical background
        \item methodologies
        \item previous studies
        \item mainstream vs. alternative viewpoints
        \item principal questions being asked
        \item general conclusions being drawn
    \end{itemize}
    \item Conclusion 
    \begin{itemize}
        \item Main agreements and disagreements of the literature
        \item any gaps or areas of futher research
        \item my overall perspective on the topic
    \end{itemize}
\end{itemize}
\medskip
Checklist for a literature review. Have I:
\begin{itemize}
    \item Outlined the purpose and scope
    \item Identified appropriate and credible (academic/scholary) literature
    \item Recorded the bibliographical details of the sources
    \item Analysed and critqued your readings
    \item Identified gaps in the readings
    \item explored methodologies / theories / hypotheses / models?
    \item discussed varying viewpoints
    \item written an intro, body and conclusion
\end{itemize}
\bigskip
Items that I want to cover in the Lit review
\begin{itemize}
    \item Using SDR (USRP) to receive GPS signal
    \item Using SDR to transmit GPS signal
    \item Algorithms for generating spoofed signal
    \item GPS Antispoofing techniques
    \item GPS spoof detection techniques
    \begin{itemize}
        \item This will be what will need to be overcome in the implmentation of the spoofing transmitter.
    \end{itemize}
    \item OPTIONAL: Difference between Block II and block III satellites
\end{itemize}
\medskip
Previous research into the use of SDR (software defined radio) for GPS spoofing uses has lead to common conclusions.
That is, that the use of open source software to ease the development. Namely multiple previous attempts at GPS spoofing
used the GNSS-SDR program for reception and the gps-sdr-sim for transmission. Both of these are available from GitHub as free
and open source programs.

\bigskip
From literature the concensus is that the same thing that has made GPS ubiquitous with navigation and 
positioning has also made it a simple target for exploitation and manipulation, that is the workings
of the inftrastructure are well known and public and are transparent and predictable \cite{RN7} \cite{RN4}. This is problematic since this infrastructure
is seen as a critical service by many industries including utility management, healthcare/ emergency services and security.
Having such a system so prone to threats is not ideal. The further developmentof SDR platforms has driven down the cost of launching
such exploits. Devices such as the HackRF, USRP and others have been documented for this use \cite{RN4} \cite{RN9}
\bigskip

\textbf{\emph{Spotr: GPS Spoofing detection via Device Fingerprinting} by M. Foruhandeh et al.}\\
As discussed in \cite{RN7} GPS (Global Positioning System) has become the defacto positioning infrastructure for civilian and 
military useage and while there are other satellite based positioning systems that are becoming more popular (GLONASS, Galileo, BeiDou) GPS is 
still the one that most people think of when thinking about GNSS. GPS provides this service globally and efficiently although through its ubiquity
security concerns have become apparent. Attacks have been carried out in both an academic and industry setting.
The authors \todo{remove reference to "authors"} believed that strengthening of the GPS framework against manipulation and exploitation must be a priority.
Proposed methods of hardening GPS have been previously propsed, however these were typically abandoned due to expense, complexity, cost or lack of robustness. 
A method of creating unique satellite "fingerprints" was developed and tested against all known spoofing attacks at time of publish.
This fingerprint algorithm was able to determine if a transmission was authentic or spoofed.
This fingerprinting algorithm implementation was found to be fast, simple, cost effective and robust when compared to previous hardening attempts. \todo{Fact check the claims of the paper} 

\medskip

\textbf{\emph{Hijacking Unmanned Aerial Vehicle by Exploiting civil GPS vulnerabilities Using SDR} by X. Zheng et al.}\\
As written by X. Zheng et al. \cite{RN4} there has been an exponential increase in the number of UAV vehicles being used in recent times.
This increase has been driven by the advent of aerial photography of both professionals and amatures since the cost of entry is so low. Ther are
also companies looking to implement delivery services using UAV, or drone, technology.
This poses security and safety concerns with such vehicles having such a reliance on GPS for accurate positioning. Having such an open and transparent
infrastructure such as that of GPS makes the signals easy to reproduce for nefarious purposes. Adding to the concern is the increase in 
affordable and performant SDR platforms as well as open source software packages designed to minimise complexity of performing such tasks. X.Zheng et al.
showed that performing spoofing attacks on UAV vehicles is cheap and simple to control. Zheng performed 3 different hijacking attacks on the target
UAV, namely forcing the drone to land, guiding the drone to an attacker dictated area and forcing the drone to land in an attacker dictated area.
Two of the attack methods were successful. Zheng also provided some recommendations for improving the security of the GPS system of, in particular
DJI Phantom 3 drones.

\medskip



%-----------------------------------
%	SUBSECTION 1
%-----------------------------------
\subsection{Subsection 1}

%-----------------------------------
%	SUBSECTION 2
%-----------------------------------

\subsection{Subsection 2}

%----------------------------------------------------------------------------------------
%	SECTION 2
%----------------------------------------------------------------------------------------

\section{Main Section 2}

