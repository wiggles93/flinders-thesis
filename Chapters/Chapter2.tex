% Chapter Template

\chapter{Literature Review} % Main chapter title

\label{Chapter2} % Change X to a consecutive number; for referencing this chapter elsewhere, use \ref{ChapterX}

%----------------------------------------------------------------------------------------
%	SECTION 1
%----------------------------------------------------------------------------------------

\section{Introduction}

In this section the previous work in the field will be explored. This will give the reader adequte understanding of where 
advances in the research area can be made.\todo{Write intro to the lit review}

\medskip
This is a list of things that need to be in a lit review:
\begin{itemize}
    \item Introduction
    \begin{itemize}
        \item Why are you writing the review, and why the topic is important
        \item the scope of the review - what aspects of the topic will be discussed
        \item the criteria used for your literature selection (types of sources used and dates etc.)
        \item the orgonisational structure of the review
    \end{itemize}
    \item Body Paragraphs
    \begin{itemize}
        \item historical background
        \item methodologies
        \item previous studies
        \item mainstream vs. alternative viewpoints
        \item principal questions being asked
        \item general conclusions being drawn
    \end{itemize}
    \item Conclusion 
    \begin{itemize}
        \item Main agreements and disagreements of the literature
        \item any gaps or areas of futher research
        \item my overall perspective on the topic
    \end{itemize}
\end{itemize}
\medskip
Checklist for a literature review. Have I:
\begin{itemize}
    \item Outlined the purpose and scope
    \item Identified appropriate and credible (academic/scholary) literature
    \item Recorded the bibliographical details of the sources
    \item Analysed and critqued your readings
    \item Identified gaps in the readings
    \item explored methodologies / theories / hypotheses / models?
    \item discussed varying viewpoints
    \item written an intro, body and conclusion
\end{itemize}
\bigskip
Items that I want to cover in the Lit review
\begin{itemize}
    \item Using SDR (USRP) to receive GPS signal
    \item Using SDR to transmit GPS signal
    \item Algorithms for generating spoofed signal
    \item GPS Antispoofing techniques
    \item GPS spoof detection techniques
    \begin{itemize}
        \item This will be what will need to be overcome in the implmentation of the spoofing transmitter.
    \end{itemize}
    \item OPTIONAL: Difference between Block II and block III satellites
\end{itemize}
\medskip
Previous research into the use of SDR (software defined radio) for GPS spoofing uses has lead to common conclusions.
That is, that the use of open source software to ease the development. Namely multiple previous attempts at GPS spoofing
used the GNSS-SDR program for reception and the gps-sdr-sim for transmission. Both of these are available from GitHub as free
and open source programs.

\bigskip

As discussed in \cite{RN7} GPS (Global Positioning System) has become the defacto positioning infrastructure
and while there are other satellite based positioning systems that are becoming more popular (GLONASS, Galileo, Beidou) GPS is 
still the one that most people think of when thinking about GNSS. GPS provides this service globally and efficiently although through its ubiquity
security concerns have become apparent. The authors \todo{remove reference to "authors"} belived that strengthening of the GPS framework against manipulation and exploitation.
Modifications to GPS have been previously propsed, however these were typically abandoned due to expense, complexity, cost or robustness. 
A method of creating unique satellite "fingerprints" was developed and tested against all known spoofing attacks at time of writing. 

%-----------------------------------
%	SUBSECTION 1
%-----------------------------------
\subsection{Subsection 1}

%-----------------------------------
%	SUBSECTION 2
%-----------------------------------

\subsection{Subsection 2}

%----------------------------------------------------------------------------------------
%	SECTION 2
%----------------------------------------------------------------------------------------

\section{Main Section 2}

