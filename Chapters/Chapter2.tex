% Chapter Template

\chapter{GNSS Systems Overview} % Main chapter title

\label{Chapter2} % Change X to a consecutive number; for referencing this chapter elsewhere, use \ref{ChapterX}

%----------------------------------------------------------------------------------------
%	SECTION 1
%----------------------------------------------------------------------------------------

\section{History of GNSS Systems}
\emph{In 2021 there are many different GNSS systems that are in place that service the globe. It is a technology that has become deeply ingrained in the everyday lives of
most people around the world. The first form of navigation system that was used by populatinons around the world was using the stats as a type of map. The first digital
based navigation system similar to what we know today was the use of terrestrial based radio transmitters. The concept was similar to that of the current satellite based
systems although the accuracy was far lower than is acceptable today. The USA developed and commissioned the first satellite based nanvigation system}

As of 2021 there are six operational GNSS systems in use. These are GPS (USA), GLONASS (Russia), Galileo (EU), BeiDou (China), IRNSS (India) and QZSS (Japan). The QZSS
system currently acts to complement the coverage of GPS in the East Asian and Oceanic regions with 4 operational satellites. There are plans to increase this number of
satellites to allow for stand alone use as a GNSS provider. 
GNSS, more specifically GPS and GLONASS, date back to 1957 during the space race with then USSR \cite{RN43}. 

When Russian satellite Sputnik 1 was launched scientists at the John Hopkins University were able to track its position and velocity by measuring the doppler shift of the
craft. This tracking from Doppler shift measurements contniued with the launch of the subsequent satellites Sputnik 2 and Explorer 1. After this it was thought that the
process could be reversed such that a satellite with a known position could be used to resolve an unknown position on earth. In the 1960's the TRANSIT satellite
navigation system was made operational for US Navy use, mainly for postion and navigation of nuclear ballistic missile submarines. TRANSIT received strong support due to
the accuracy of up to 80ft (24m) which was a significant improvement over existing VLF (Very Low Frequency) hyperbolic navigation systems. After 32 years of operation the
system was retired in 1996 after proving that space crafts could be reliable \cite{RN45}. 

In 1973 the US combined two existing programmes in TIMATION and 'Project 621B' to form the 'NAVSTAR Global Positioning System' which would later become known more
commonly as GPS. The initial intention for the use of GPS was for military only. However, in 1983 there was an incident that saw a Korean Airlines flight shot down by a
Soviet fighter mistaking it for a US aircraft when it wandered off course. After this incident President Reagen announced that GPS would be made available for civilian
use. The military insisted that the accuracy of the system be purposley degraded for civilian use though selective availabililty such that GPS could not be used by
adversaries. In 1993 the system was declared operational. To combat the intentional accuracy reduction of the GPS system augmentation systems started to appear. Soon
after there were Government funded versions of DGPS (Differetial GPS) systems \cite{RN43}. The significantly increases the accuracy by using a receiver with a knwon
location and having that reciever calculate its position from the satellite signals and compare with the known position. The corrections are then broadcast. 

The soviet union launched their first GLONASS satellite in 1982, and in the following three years 10 more were launched. There were some technical differences between the
GLONASS and GPS constellations. The orbital planes were different and GLONASS uses an FDMA (Frequency Division Multiple Access) scheme as opposed to the GPS CDMA (Code
Division Multiple Access). In 1993 president Yeltsin declared that the GLONASS constllation was fully operational, however this was not the case. 


\todo{add graph: x axis year y axis accuracy} 

\section{GNSS Systems}
\subsection{Introduction to GNSS Constellations}
GNSS is a term used to describe any satllelite system that provides positon and timing information. All GNSS systems in use today have three sepearte segements that
encompass the term GNSS. That is the ground segment, space segment and user segment. These are used in conjunction to allow the user to calculate with accuracy up to $\pm
10cm$. In order to calculate an unknown poistion on earth, the exact position of the orbiting satellite must be known. The satellite will send it's exact position as well
as it's current time down to the earths surface. Receviers (the user segment) will pick up this signal. The components of the message sent are quite simple and consist
mostly of the current time (of the satellite) and where the sattelite is in the WGS-84 coordinate system\cite{RN46}. This coordinate system has its origin at the centre of mass of
the earth with the Z axis pointing at the north pole, the X axis pointing towards the prime meridian and the y axis perpendicular to both other axes. Due to the reletive
motion of the satllite and the reciever the transmitted signal will be shifted up or down the frequency range due to the doppler effect. The combination of this doppler
shift, time taken to arrive and the satellite current location can be used to solve the postiion in three dimensional space. In order to solve for all dimensions
(including time) four satellites are reqruied since there are essentially 4 unknown terms to solve for as shown in equation \ref{eq:gnssPosition} \cite{RN46}. The
$\rho_i$ term indicates the distance from the recviever to that particular satellite. This method of position calculation is known as trilateration. The model shown below
is a simplified model that ignores the effects of relativity. 

\begin{equation} \label{eq:gnssPosition} 
    \rho_i = \sqrt{(x_i - x_u)^2 + (y_i - y_u)^2 + (z_i - z_u)^2} + c \Delta t
\end{equation}
\todo{look at RN32 for position and time calculations}
As can be seen from the above equation \ref{eq:gnssPosition} the accurate synchronisation of time betweeen the receiver and the satellites is required in order to
calcaulte an accurate position. Due to the mathmatical requirements, a minimum of 4 satellites need to be in 'sight' of the receiver at all times.

Each of the operational constellations use the same fundamentals to provide accurate postion data, but each differ in some key aspects as discussed below.

\subsection{GPS}
The GPS constellation consists of a nominal 24 satelelites in 6 orbital planes to ensure that there is coverage globally at all times.  

\subsection{Galileo}

\subsection{GLONASS}
The GLONASS constelaltion also consits of a nominal 24 satellites, although only in 3 orbital planes. Galileo has 30 satlellites also with 3 orbital planes.

\subsection{BeiDou}

\subsection{Other Constellations}

\section{Applications of GNSS}
GNSS applications are broad and deeply ingradined in the almost everyones everyday lives. These applications vary from providing postion and navigation information for
maps on a smartphone or sat-nav devce to providing accurtate timing infomation for financial institutions \cite{RN33}. 