% Chapter Template

\chapter{Introduction to GNSS Systems} % Main chapter title

\label{Chapter2} % Change X to a consecutive number; for referencing this chapter elsewhere, use \ref{ChapterX}

%----------------------------------------------------------------------------------------
%	SECTION 1
%----------------------------------------------------------------------------------------

\section{History of GNSS Systems}
\emph{In 2021 there are many different GNSS systems that are in place that service the globe. It is a technology that has become deeply ingrained in the everyday lives of
most people around the world. The first form of navigation system that was used by populatinons around the world was using the stats as a type of map. The first digital
based navigation system similar to what we know today was the use of terrestrial based radio transmitters. The concept was similar to that of the current satellite based
systems although the accuracy was far lower than is acceptable today. The USA developed and commissioned the first satellite based nanvigation system}

As of 2021 there are six operational GNSS systems in use. These are GPS (USA), GLONASS (Russia), Galileo (EU), BeiDou (China), IRNSS (India) and QZSS (Japan). The QZSS
system currently acts to complement the coverage of GPS in the East Asian and Oceanic regions with 4 operational satellites. There are plans to increase this number of
satellites to allow for stand alone use as a GNSS provider. 
GNSS, more specifically GPS, dates back to 1957 during the space race with then USSR \cite{RN43}. 

When Russian satellite Sputnik 1 was launched scientists at the John Hopkins University were able to
track its position and velocity by measuring the doppler shift of the craft. This tracking from Doppler shift measurements contniued with the launch of the subsequent
satellites Sputnik 2 and Explorer 1. After this it was thought that the process could be reversed such that a satellite with a known position could be used to resolve an
unknown position on earth. In the 1960's the TRANSIT satellite navigation system was made operational for US Navy use, mainly for postion and navigation of nuclear
ballistic missile submarines. TRANSIT received strong support due to the accuracy of up to 80ft (24m) which was a significant improvement over existing VLF (Very Low
Frequency) hyperbolic navigation systems. After 32 years of operation the system was retired in 1996 after proving that space crafts could be reliable \cite{RN45}. 

In 1973 the US combined two existing programmes in TIMATION and 'Project 621B' to form the 'NAVSTAR Global Positioning System' which would later become known more
commonly as GPS. The initial intention for the use of GPS was for military only. However, in 1983 there was an incident that saw a Korean Airlines flight shot down by a
Soviet fighter mistaking it for a US aircraft when it wandered off course. After this incident President Reagen announced that GPS would be made available for civilian
use. The military insisted that the accuracy of the system be purposley degraded for civilian use though selective availabililty such that GPS could not be used by
adversaries. In 1993 the system was declared operational. To combat the intentional accuracy reduction of the GPS system augmentation systems started to appear. Soon
after there were Government funded versions of DGPS (Differetial GPS) systems \cite{RN43}. The significantly increases the accuracy by using a receiver with a knwon location and
having that reciever calculate its position from the satellite signals and compare with the known position. The corrections are then broadcast. 

The soviet union launched their first GLONASS satellite in 1982, and in the following three years 10 more were launched. There were some technical differences between the
GLONASS and GPS constellations. The orbital planes were different and GLONASS uses an FDMA (Frequency Division Multiple Access) scheme as opposed to the GPS CDMA (Code
Division Multiple Access). In 1993 president Yeltsin declared that the GLONASS constllation was fully operational, however this was not the case. 


\todo{add graph: x axis year y axis accuracy} 