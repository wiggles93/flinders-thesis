% Chapter Template

\chapter{Introduction}\label{chapter:firstchapter} % Main chapter title

\label{Chapter1} % Change X to a consecutive number; for referencing this chapter elsewhere, use \ref{ChapterX}

%----------------------------------------------------------------------------------------
%	SECTION 1
%----------------------------------------------------------------------------------------
\section{Motivation}\label{sec:Motivation}

% It is a good idea to have each sentence on a separate line, so that if you get feedback or changes from someone else
% the diffs will be much easier to manage

As society moves through the age of technology there is an exponential reliance on reliable access to position and time data. The main source of this infomation over the
recennt decades has been through GNSS constellations. GNSS services are now tightly integrated with many facets of life from perosnal navigastion, public transport and
management of
energy infrasture. This has made these services the target of attacks. In order to provide adequete defence knowledge of how the attacks are performed is requiered.

\section{Aims and Benefits}\label{sec:Aims}
The aims of this thesis are to investigate and perform a GPS spoofing attack in order to gain a better understanding of the operation of the GPS infrastructure and attack
methodology. Emphasis will be placed on the technical knowledge requried to perform such attacks and the cost of performing them now as opposed to the past. This will allow for further work in counter-measures of GPS spoofing.

\section{Research Questions}\label{sec:RQs}
\begin{enumerate}
    \item What resources are required to perform a spoofing attack on commercial GPS receiver?
    \item What effort and technical expertise is required for a spoofing attack on a commercial GPS receiver?
    \item How can a prototype GPS spoofer be implemented in a controlled environment?
    \item How could a prototype be implemented in a real-time static or dynamic environment? 
\end{enumerate}

\section{Limitations}\label{sec:Limits}
Due to hardware and software limitations, all testing will be performed on the Navstar GPS system only. There will be no multi constellation. However the theory and basic
structure of the attacks are relevant to all constellations since the same trilateration technique is common to them all.

\section{Thesis Structure}\label{sec:structure}
The rest of this thesis will have a structure as follows, Chapter \ref{Chapter2} will introduce the history and concepts in GNSS technology as well as defining and
introducing GNSS spoofing. Chapter \ref{Chapter3} provides a summary of relevant research in the area of GNSS spoofing attacks and defences. Spoof defence will include
detection and mitigation. Chapter \ref{Chapter4} outlines the method used in the spoofing attack. Chapter \ref{Chapter5} will show the results gathered from the
experimentation. Chapter \ref{Chapter6} will provide discussion regarding the results of the experiments as well as issues and solutions encoutered while performing the
experiments as well as recommendations for future work. Finally chapter \ref{Chapter7} willl provide a conclusion to the project.