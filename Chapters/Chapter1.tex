% Chapter Template

\chapter{Introduction}\label{chapter:firstchapter} % Main chapter title

\label{Chapter1} % Change X to a consecutive number; for referencing this chapter elsewhere, use \ref{ChapterX}

%----------------------------------------------------------------------------------------
%	SECTION 1
%----------------------------------------------------------------------------------------
\section{Motivation}\label{sec:Motivation}

% It is a good idea to have each sentence on a separate line, so that if you get feedback or changes from someone else
% the diffs will be much easier to manage

As society moves through the age of technology there is an exponential reliance on reliable access to position and time data. The main source of this information over the
recent decades has been through GNSS constellations. GNSS services are now tightly integrated with many facets of life from personal navigation, public transport and
management of energy infrastructure. This has made these services the target of attacks. In order to provide adequate defence knowledge of attack methodologies are
required. 
There has been more time and research put into the defence of spoofing attacks, and not so into the method of performing a successful spoofing attack or
circumventing existing anti-spoofing methods. 

As it currently stands there is no active research or efforts within GNSS attacks or EW (Electronic Warfare) in general from within \univname. Therefore, to ensure that
\univname  is able to keep up with moving trends and
ever advancing technology an EW chair was devised with this project being the first completed in conjunction with. GNSS technology provides critical information for
civilian purposes as well as in defence and military. There is a constant need to be aware of the latest research in such ares.   
This thesis will document background information on both GNSS operating principles as well as background research in
spoofing. A workflow was generated and will be used as a starting point for continual research. This will set the tone for further research into EW.

\section{Aims and Benefits}\label{sec:Aims}
The aims of this thesis were to investigate and perform GPS spoofing attacks in order to gain a better understanding of the operation of the GPS infrastructure and attack
methodology and to create hardware capable of performing a spoofing attack. This will be used to create a baseline for SDR capabilities research from within \univname where future research can build upon this initial documentation.
The depth of research into the most cutting edge SDR and GPS Spoofing technologies will therefore not be covered in this paper but will be the topic of future research
within the College of Science and Engineering at \univname.
There was  emphasis placed on the technical knowledge required to perform such attacks and the cost of performing them now as opposed to the past. This will
allow for further work in counter-measures of GPS spoofing.
This thesis will purely concentrate on the research of and implementation of a spoofing device. There will be no implementation of anti-spoofing methodology.

\section{Research Questions}\label{sec:RQs}
\begin{enumerate}
    \item What hardware is required to perform a spoofing attack on commercial GPS receiver?
    \item What technical expertise is required for a spoofing attack on a commercial GPS receiver?
    \item How can a prototype GPS spoofer be implemented in a controlled environment?
    \item How could a prototype be implemented in a real-time static or dynamic environment? 
\end{enumerate}

\section{Limitations}\label{sec:Limits}
Due to hardware and software limitations, all testing was performed on the Navstar GPS system only. There was no multi constellation attacks. However, the theory and basic
structure of the attacks are relevant to all constellations since the same trilateration technique is common to them all.

Since GPS and more specifically the L1 frequency band of GPS is so ubiquitous in society there is open source software available that made this project achievable. While
it would be possible to extend the project to encompass other constellations and frequency bands, this would require much more time that is allowed for.

\section{Thesis Structure}\label{sec:structure}
The rest of this thesis will have a structure as follows, Chapter \ref{Chapter2} will introduce the history and concepts in GNSS technology as well as defining and
introducing GNSS spoofing. Chapter \ref{Chapter3} provides a summary of relevant research in the area of GNSS spoofing attacks and defences. Spoof defence will include
detection and mitigation. Chapter \ref{Chapter4} outlines the method used in the spoofing attack. Chapter \ref{Chapter5} will show the results gathered from the
experimentation. Chapter \ref{Chapter6} will provide discussion regarding the results of the experiments as well as issues and solutions encountered while performing the
experiments as well as recommendations for future work. Finally chapter \ref{Chapter7} will provide a conclusion to the project.